\documentclass[11pt]{article}  
\iffalse
--------------------------------------------

To do:
JMLR logins: peter.bartlett, stevenneilevans
      

To consider:

What about boosting smoothness through depth in more generality --
Sobolev and/or Besov spaces?  

What about not invertible? e.g., lower output dimension?

Neater analysis of ||h-h*||.

Dispensing with differentiability of $h_i$: denseness.

Is smoothness of h necessary?

Lower bounds on best decomposition? Especially in terms of
alpha,M,R,...

What about loss other than quadratic?

Optimization? What does lower bound on gradient say about progress?

Algorithms? Constructive algorithms in each layer.
\fi
\usepackage{amsmath, amssymb, amsthm}  
\usepackage{fullpage}  
\usepackage{mathtools}  
\usepackage{enumitem}  
\usepackage{bm}  
\usepackage{float}     % provides [H]
\usepackage{graphicx}

\title{Orthogonal Skip Connections}  
\author{Filip Morawiec}  
\date{June 2025}
  
\newtheorem{theorem}{Theorem}  
\newtheorem{lemma}{Lemma}  
  
\newcommand{\norm}[1]{\left\|#1\right\|}  
\newcommand\calX{{\mathcal{X}}}  
  
\begin{document}  
\maketitle  
  
\begin{abstract}  
   %  Inspired by how ResNet addressed the problem of exploding gradients by skip connections, we explore how replacing skip connections with orthogonal weights affects training of the network.  
   ResNet architecture was a breakthrough in deep learning. 
   Its success has been attributed to the use of skip connections, which in turn mitigated the problem of exploding gradients.
   It has been shown that ResNet overcame exploding gradients because it's skip connections (identity matrices) are norm preserving. 
   In this paper, we replace these identity matrices with orthogonal matrices, which are by defintion norm preserving. 
   First, following the previous formulation of ResNet, we interestingly show that as the number of layers grows, the orthogonal matrices need to converge to the identity matrix. 
   Then, we explore the performance of orthogonal skip connections across multiple regression tasks, using the recent efficient algorithm for gradient descent on orthogonal matrices~\cite{modula-docs}. 
   In contrary to previous works, we show that orthogonal skip connections do not outperform identity skip connections and require more training time to converge.
   We also analyze the validity of the theoretical construction, showing that the orthogonal matrices do not converge to identity matrices in practice as fast as the theory suggests.
   Finally, we explore alternative training schemes that could make use of orthogonal skip connections, but find that they do not outperform the standard training scheme. \textit{TODO: apply different training schemes, optimizers.}
\end{abstract}  
  
\tableofcontents
\vspace{1cm}

\hrule

\newpage

\section{Introduction}

Our main contribtions:
\begin{itemize}
   \item we show that under a certain mathematical construction, orthogonal skip connections converge to identity connections (as network depth increases);
   \item we explore if the theoretical construction actually matches the trained model;
   \item we explore alternative schemes of trainings that could make use of orthogonal skip connections;
   \item we argue that orthogonal skip connections aren't a good replacemenet for identity skip connections
\end{itemize}

\section{Related works}

This paper has already done it, but they claim the method works, but it is 1 percentage point better: https://arxiv.org/abs/1707.05974

Recent paper with a different idea: https://arxiv.org/abs/2505.11881.


\section{Theoretical considerations}

This part wouldn't have been possible if not the work of \cite{bartlett2018representingsmoothfunctionscompositions}, 
which came up with a construction of the decomposition of a function into near-identity functions. Their proof has been adapted to orthogonal matrices. 

\subsection{Notation and definitions}

\subsection{Decomposing a function into near-orthogonal functions}  
  
\begin{theorem}\label{theorem:main1}  
Fix $R>0$, denote $\calX=B_R(\mathbb{R}^d)$.  
Let $h:\mathbb{R}^d\to\mathbb{R}^d$ be a differentiable, invertible map satisfying  
the following properties:  
\begin{enumerate}  
    \item \textbf{Smoothness:} for some $\alpha>0$  
    and all $x,y,u\in\calX$,  
     \begin{equation}  
      \label{eq:smoothness}  
        \left\|\bigl(D h(y) - D h(x)\bigr)u\right\|  
           \le \alpha\|y-x\|\|u\|;  
     \end{equation}  
     \item \textbf{Lipschitz inverse:}  
     for some $M>0$, $ \|h^{-1}\|_L\le M$;  
     \item \textbf{Positive orientation:} For some $x_0\in\calX$,  
    $\det\bigl(Dh(x_0)\bigr)>0$.  
   %  h(0) = 0
   \item \textbf{Zero at origin:} $h(0) = 0$
\end{enumerate}  
  
Then for all $m$, there exist $m$ functions  
$h_1,\ldots,h_m:\mathbb{R}^d\to\mathbb{R}^d$ and $m$ orthogonal matrices $W_1, \ldots W_n \in \mathbb{R}^{n \times n}$ satisfying,  
for all $x\in\calX$,  
\[
    h_m\circ h_{m-1}\circ\cdots\circ h_1(x) \;=\; h(x),  
\]
and, on $h_{i-1}\circ\cdots \circ h_1(\calX)$, the following terms decrease in $O\left(\frac{\log m}{m}\right)$ rate:  $\|h_i-W_i\|_L, \ \| I - W_i\|$ for $i=1,\ldots,m$.

\textbf{TODO: We can probably replace the $\log{m}/m$ term by choosing any sequence in $(0, 1)$ that converges to $0$}

\end{theorem}  
  
\vspace{1cm}  
  
To help prove the theorem, we will note the following:  
  
\begin{lemma}\label{lemma:smoothnessimplications}  
    For $h$ satisfying   
    the conditions of Theorem~\ref{theorem:main1}  
    and any $x,y\in\calX$,  
      \begin{align*}  
        \bigl\|h(y) - h(x)- Dh(x)(y-x)\bigr\|   
          & \;\le\; \frac{\alpha}{2}\|y-x\|^2,  
      \end{align*}  
    and moreover,  
\[
         \|h\|_L \;\le\; 1+\alpha R.  
\]
\end{lemma}  
  
Now we can proceed to prove Theorem~\ref{theorem:main1}. 
  
\begin{proof}  
  
To organize the proof, we will first set up our notation and then prove two main bounds (Case~1 and Case~2) that together imply the final estimate.  
  
\subsubsection{Proof setup}  
  
For $i = 1, 2, \ldots, m$, define  
\begin{equation*}  
    g_i(x) \;:=\; \frac{1}{a_i}\;W_i^\top\,h\bigl(a_i\,W_i\,x\bigr),\,g_i : \mathcal{X} \longrightarrow \mathbb{R}^d.
\end{equation*}  

Each $a_i>0$ is a scalar parameter with
\[
   0 < a_1 < a_2 < \cdots < a_m = 1, \ a_i = \left(1 - c\right)^{m-i} \text{ for some } c \in (0,1).
\]

The orthogonal matrices $W_i\in \mathbb{R}^{d \times d}$ satisfy $\| I - W_i \| \leq C \| a_i - a_{i-1}\|$ for some constant $C>0$ and all $i=1,\ldots,m$.
Note that from this property we obtain bounds on closeness of consecutive $W_i$'s:
\begin{align} 
   \| W_i - W_{i-1} \| &\leq \| I - W_i \| + \| I - W_{i-1} \| \\
   &\leq C \| a_i - a_{i-1} \| + C \| a_{i-1} - a_{i-2}\| \\
   &\leq C \| a_i - a_{i-2} \| < C \| a_i - a_{i-1}\| \label{eq:W_i-W_i-1}
\end{align}
and
\begin{align} 
   \| a_i W_i - a_{i-1} W_{i-1} \| &\leq \| a_i W_i - a_i W_{i-1} \| + \| a_i W_{i-1} - a_{i-1} W_{i-1} \| \\
   &\leq a_i \| W_i - W_{i-1} \| + \| a_i - a_{i-1} \| \\
   &< a_i  C \|a_i - a_{i-1}\| + \| a_i - a_{i-1} \|  = \| a_i - a_{i-1} \| (C a_i + 1)  \label{eq:W_i-W_i-1-2}
\end{align}


Then we set 
\[
   h_i \;:=\; g_i \,\circ\, g_{i-1}^{-1}\quad (i=1,\ldots,m),
\]
with the convention $g_0 = \mathrm{Id}$ (the identity map).  Observe that 
\[
   h \;=\; h_m \circ \cdots \circ h_1 
   \;=\; g_m\circ g_{m-1}^{-1}\; \circ\; \cdots\; \circ g_1\circ g_0^{-1}
   \;=\; g_m \circ g_{m-1} \circ \cdots \circ g_1,
\]
so indeed the composition of $h_1,\dots,h_m$ yields $h$.  

We aim to show 
\[
   \|h_i - W_i\|_L 
   \;:=\; \sup_{x \neq y}\,
          \frac{\bigl\|\bigl(h_i(x) - W_i x\bigr) - \bigl(h_i(y) - W_i y\bigr)\bigr\|}
               {\|x - y\|}  
   \;\; \text{is}\; O\!\bigl(\tfrac{\log m}{m}\bigr).
\]

\subsubsection{Bounding the first term $\|h_1 - W_1\|_L$}  

We have
\[
   h_1(x)
   \;=\;
   g_1 \bigl(x\bigr)
   \;=\;
   \frac{1}{a_1}\,W_1^\top\,h\bigl(a_1\,W_1\,x\bigr).
\]
Hence
\[
   h_1(x) - W_1\,x
   \;=\;
   \frac{1}{a_1}\,W_1^\top\,h\bigl(a_1\,W_1\,x\bigr)
   \;-\; W_1\,x.
\]
Consider two inputs $x,y\in \calX$.  We want to bound
\[
   \bigl\|\bigl(h_1(x)-W_1x\bigr) - \bigl(h_1(y)-W_1y\bigr)\bigr\|.
\]
Using Lemma~\ref{lemma:smoothnessimplications}, expand
\[
  h\bigl(a_1\,W_1\,x\bigr) 
  \;=\; 
  h\bigl(a_1\,W_1\,y\bigr) 
  \;+\; Dh\bigl(a_1\,W_1\,y\bigr)\,\bigl(a_1\,W_1\,(x-y)\bigr) 
  \;+\; r,
\]
where $\|\,r\,\|\le (\alpha/2)\,\|\,a_1\,W_1\,(x-y)\|^2$.  Hence
\begin{align*}
  &\bigl\|\bigl(h_1(x)-W_1x\bigr) - \bigl(h_1(y)-W_1y\bigr)\bigr\|
  = \\
  &\biggl\|\;
  \frac{1}{a_1}\,W_1^\top\,\bigl[h\bigl(a_1\,W_1\,x\bigr)
      - h\bigl(a_1\,W_1\,y\bigr)\bigr]
  \;-\; W_1\,(x-y) 
  \biggr\| \leq \\
     &\biggl\|\;\frac{1}{a_1}\,W_1^\top\,Dh\bigl(a_1\,W_1\,y\bigr)\,\bigl(a_1\,W_1\,(x-y)\bigr) - W_1\,(x-y)\biggr\| + \biggl\|\;\frac{1}{a_1}\,W_1^\top\,r\biggr\| \leq \\
     &\biggl\| Dh(W_1 y) - I \biggr\|  a_1 \left\| x - y \right\|  + \frac{\alpha a_1}{2} \left\| x - y \right\|^2 = \\ &a_1 \left\| x - y \right\| \left[ \biggl\| Dh(W_1 y) - I \biggr\| - \frac{\alpha}{2} \left\| x - y \right\| \right]
\end{align*}
    
\subsubsection{Bounding the general term $\|h_i - W_i\|_L$}  
  
For $i>1$, recall 
\[
   h_i \;=\; g_i \circ g_{i-1}^{-1},
\]
so
\[
   h_i(x) - W_i\,x 
   \;=\; g_i\bigl(g_{i-1}^{-1}(x)\bigr)\;-\;W_i\,x.
\]
Given $x,y\in \calX$, define 
\[
  u := g_{i-1}^{-1}(x), 
  \quad
  v := g_{i-1}^{-1}(y).
\]
Then 
\[
   x = g_{i-1}(u), 
   \quad 
   y = g_{i-1}(v).
\]
Note that since $\| h^{-1} \|_L \leq M$, we have 

\begin{align*}
   \left\| x - y \right\| = \frac{1}{a_{i-1}} &\left\| a_{i-1} W_{i-1} g_{i-1}(u) - a_{i-1} W_{i-1} g_{i-1}(v) \right\| \geq \\
   \frac{1}{M a_{i-1}} &\left\| h^{-1}(a_{i-1} W_{i-1} g_{i-1}(u)) - h^{-1}( a_{i-1} W_{i-1} g_{i-1}(v)) \right\| = \\
   \frac{1}{M a_{i-1}} &\left\| h^{-1}(h\bigl(a_{i-1}\,W_{i-1}\,u\bigr)) - h^{-1}(h\bigl(a_{i-1}\,W_{i-1}\,v\bigr)) \right\| = \\
   \frac{1}{M a_{i-1}} &\left\| a_{i-1}\,W_{i-1}\,u - a_{i-1}\,W_{i-1}\,v \right\| = \\
   \frac{1}{M} &\left\| u - v \right\|
\end{align*}

and also $\| v \| \leq M R$ and $\| u \| \leq M R$. \textbf{TODO: Update the following to use this}


The difference we need to bound is
\[
   \bigl\|\bigl(h_i(x) - W_i x\bigr) - \bigl(h_i(y) - W_i y\bigr)\bigr\| 
   \;=\;
   \bigl\|\,g_i(u) - g_i(v) - W_i\bigl[g_{i-1}(u) - g_{i-1}(v)\bigr]\bigr\|.
\]
Recall
\[   
   g_i(u)
   \;=\;
   \frac{1}{a_i}\,W_i^\top\,h\bigl(a_i\,W_i\,u\bigr),
   \quad
   g_{i-1}(u)
   \;=\;
   \frac{1}{a_{i-1}}\,W_{i-1}^\top\,h\bigl(a_{i-1}\,W_{i-1}\,u\bigr).
\]
Hence we can rewrite
\begin{align*}
  &\;\;\;\;g_i(u) - g_i(v) - W_i\bigl[g_{i-1}(u) - g_{i-1}(v)\bigr]
  \\
  &=  
  \frac{1}{a_i}\,W_i^\top\,\Bigl[h\bigl(a_i\,W_i\,u\bigr)-h\bigl(a_i\,W_i\,v\bigr)\Bigr]
  \;-\;  
  \frac{1}{a_{i-1}}\,W_i\,W_{i-1}^\top\,\Bigl[h\bigl(a_{i-1}\,W_{i-1}\,u\bigr)-h\bigl(a_{i-1}\,W_{i-1}\,v\bigr)\Bigr].
\end{align*}
To shorten notation, let
\[
   A := a_i\,W_i\,u,\quad B:= a_i\,W_i\,v,
   \quad
   X := a_{i-1}\,W_{i-1}\,u,\quad Y := a_{i-1}\,W_{i-1}\,v.
\]
By Lemma~\ref{lemma:smoothnessimplications}, we can write
\begin{align*}
   h(A) - h(B) 
   &= Dh(B)\,(A-B) + r_1,
   \quad \text{where } \|r_1\|\le \tfrac{\alpha}{2}\|A-B\|^2,  \\
   h(X) - h(Y) 
   &= Dh(Y)\,(X-Y) + r_2,
   \quad \text{where } \|r_2\|\le \tfrac{\alpha}{2}\|X-Y\|^2.
\end{align*}
and expanding further:
\begin{align*}
   \|r_1\|\le \tfrac{\alpha}{2}\|A-B\|^2 &\leq \tfrac{\alpha}{2} a_{i}^2 \| u - v \| ^ 2 \\
   \|r_2\|\le \tfrac{\alpha}{2}\|X-Y\|^2 &\leq \tfrac{\alpha}{2} a_{i-1}^2 \| x - y \| ^ 2 
\end{align*}

Thus
\begin{align*}
  &\Bigl\|\tfrac{1}{a_i}\,W_i^\top\bigl[h(A) - h(B)\bigr]
    \;-\; \tfrac{1}{a_{i-1}}\,W_i\,W_{i-1}^\top\bigl[h(X)-h(Y)\bigr]\Bigr\|
  \\
  &=  
  \Bigl\|\,
   \tfrac{1}{a_i}\,W_i^\top\bigl[Dh(B)\,(A-B) + r_1\bigr]
   \;-\;
   \tfrac{1}{a_{i-1}}\,W_i\,W_{i-1}^\top\bigl[Dh(Y)\,(X-Y) + r_2\bigr]
  \Bigr\|
  \\
  &\le
    \Bigl\|\,
    \tfrac{1}{a_i}\,W_i^\top\,Dh(B)\,(A-B)
    \;-\;
    \tfrac{1}{a_{i-1}}\,W_i\,W_{i-1}^\top\,Dh(Y)\,(X-Y)
    \Bigr\|
  \;+\;
  \underbrace{
    \Bigl\|\,
    \tfrac{1}{a_i}W_i^\top\,r_1 \;-\;
    \tfrac{1}{a_{i-1}}\,W_i\,W_{i-1}^\top\,r_2
    \Bigr\|
  }_{r \, := } \\ 
  &\leq \Bigl\|\,\underbrace{
    \tfrac{1}{a_i}\,W_i^\top\,Dh(B)\,a_i\,W_i
    \;-\;
    \tfrac{1}{a_{i-1}}\,W_i\,W_{i-1}^\top\,Dh(Y)\,a_{i-1}\,W_{i-1}
    }_{J \, := }\Bigr\| \Bigl\| u - v \Bigr\|\,+\,r.
\end{align*}

We can split and bound the term $J$ in the following way:

\begin{align*}
    \|J\| &= \| \tfrac{1}{a_i}\,W_i^\top\,Dh(B)\,a_i\,W_i
    \;-\;
    \tfrac{1}{a_{i-1}}\,W_i\,W_{i-1}^\top\,Dh(Y)\,a_{i-1}\,W_{i-1} \| = \\
    &= \| \,W_i^\top\,Dh(B)\,W_i
    \;-\;
    \,W_i\,W_{i-1}^\top\,Dh(Y)\,W_{i-1} \| = \\
    &= \left\| \left[ \,W_i^\top\,Dh(B)\,W_i - \,W_i^\top\,Dh(Y)\,W_i \right] + \left[ \,W_i^\top\,Dh(Y)\,W_i-\,W_i\,W_{i-1}^\top\,Dh(Y)\,W_{i-1} \right] \right\| \leq \\
    &\leq \alpha \bigl\| B - Y\bigl\|^2 + \bigl\| \underbrace{\,W_i^\top\,Dh(Y)\,W_i-\,W_i\,W_{i-1}^\top\,Dh(Y)\,W_{i-1} }_{J_2 \ :=}\bigr\|
\end{align*}

Then, abusing notation and writing $Dh(Y)$ as $D$
\begin{align*}
    \left\| J_2\right\| &= \left\| \left[ W_i^\top\,D\,W_i - W_i^\top\,D\,W_{i-1} \right] + \left[W_i^\top\,D\,W_{i-1} - \,W_i\,W_{i-1}^\top\,D\,W_{i-1} \right] \right\| \leq \\
    &\leq \left\| D \right\| \| W_i - W_{i-1} \| + \left\| W_i^\top - \,W_i\,W_{i-1}^\top \right\| \left\|  D \right\| = \\
    &= \left\| D \right\| \left( \| W_i - W_{i-1} \| + \left\| W_i^\top - W_i W_i^\top + W_i W_i^\top - \,W_i\,W_{i-1}^\top\right\| \right) \leq \\
    &\leq \left\| D \right\| \left( \| W_i - W_{i-1} \| + \left\| W_i^\top - W_i W_i^\top \right\| + \left\| W_i W_i^\top - \,W_i\,W_{i-1}^\top\right\| \right) = \\
    &= \left\| D \right\| \left( \| W_i - W_{i-1} \| + \left\| I - W_i \right\| + \left\| W_i^\top - \,W_{i-1}^\top\right\| \right) = \\
    &= \left\| D \right\| \left\| \left( \| W_i - W_{i-1} \| + \left\| I - W_i \right\| + \left\| W_i - \,W_{i-1}\right\| \right) \right\|.
\end{align*}

Note that we can bound $D$ by:
\begin{equation}
   \| Dh(a_{i-1} W_{i-1} v) \| \leq \alpha \left\| a_{i-1} W_{i-1} v \right\| \leq \alpha a_{i-1} M R
\end{equation}

Combining, we obtain:

\begin{align*}
   \bigl\|\bigl(h_i(x) - W_i x\bigr) - \bigl(h_i(y) - W_i y\bigr)\bigr\|  
   \leq r\,&+\,\left\| u - v \right\| \left( \alpha \left\| B - Y \right\|^2  + J_2 \right)
   \leq \\
   \leq 
   \tfrac{\alpha}{2} \left( a_{i}^2 \| u - v \| ^ 2 +  a_{i-1}^2 \| x - y \| ^ 2 \right)
      &\,+\,\\
      \left\| u- v \right\| 
      \bigl( 
         \alpha \left\| a_i W_i v - a_{i-1} W_{i-1} v \right\|^2
         &+\left\| a_{i-1} M R \right\| \left( 2 \| W_i - W_{i-1} \| + \left\| I - W_i \right\| \right) 
      \bigr) \\
   \leq \tfrac{\alpha}{2} \left( (a_{i}^2 M^2 +  a_{i-1}^2) \| x - y \| ^ 2 \right)
   &\,+\,\\
   M \left\| x - y \right\| 
   \bigl( 
      \alpha \left\| a_i W_i - a_{i-1} W_{i-1} \right\| R^2
      &+ \left\| a_{i-1} M R \right\| \left( 2\| W_i - W_{i-1} \| + \left\| I - W_i \right\| \right)
   \bigr) \\
   \leq \| x - y \| \bigl[ 
      \tfrac{\alpha}{2} M^2 \| x - y \|
      +  \alpha \left\| a_i W_i - a_{i-1} W_{i-1} \right\| R^2
      &+\left\| a_{i-1} M R \right\| \left( 2\| W_i - W_{i-1} \| + \left\| I - W_i \right\| \right)
   \bigr] \\
   \text{using the equations \eqref{eq:W_i-W_i-1} and \eqref{eq:W_i-W_i-1-2}, we can write: } &\\
   < \| x - y \| \bigl[ 
      \tfrac{\alpha}{2} M^2 \| x - y \|
      +  \alpha (C a_i + 1) (a_i - a_{i-1}) R^2
      &+\left\| a_{i-1} M R \right\| \left( 2C (a_i - a_{i-1}) + C (a_i - a_{i-1}) \right) \bigr] \\
   \leq \| x - y \| \bigl[
      \tfrac{\alpha}{2} M^2 \| x - y \|
      +  \alpha (C a_i + 1) (a_i - a_{i-1}) R^2
      &+3C \left\| a_{i-1} M R \right\| (a_i - a_{i-1}) \bigr]. 
\end{align*}

\paragraph{Case 1: Arguments are close.}

% TODO: is this optimal? necessary?
If $\| x - y \| \leq \| a_i - a_{i-1} \|$, then we can bound the above expression by

\begin{align*}
   \| h_i - W_i \|_L &< \bigl[ 
      \tfrac{\alpha}{2} M^2 +  \alpha (C a_i + 1) R^2
      +3C \left\| a_{i-1} M R \right\|  \bigr] \| a_i - a_{i-1} \| \\
\end{align*}

\paragraph{Case 2: Arguments are far.} 

If $\|x-y\| \geq a_i - a_{i-1}$, we can again use the fact that the increments $a_i - a_{i-1}$, as well as the orthogonal matrix differences $\|W_i - W_{i-1}\|$ and $\|I - W_i\|$, are all $O(a_i - a_{i-1})$, but this time also use the global Lipschitz properties of $h$.

We can rewrite the main term, substituting $S_k(z) = g_k(z) - g_{k-1}(z)$:
\begin{align*}
   \bigl(h_i(x) - W_i x\bigr) - \bigl(h_i(y) - W_i y\bigr) &= 
      \left[g_i(u) - g_{i-1}(u) - (g_i(v) - g_{i-1}(v))\right]
      + (I - W_i)\bigl[g_{i-1}(u) - g_{i-1}(v)\bigr] \\
      &= S_i(u) + S_i(v) + (I - W_i)\bigl[g_{i-1}(u) - g_{i-1}(v)\bigr] = \\
      &= S_i(u) - S_i(v) + (I - W_i) \bigl[x - y\bigr]
\end{align*}

We have already shown the term $I - W_i = O(a_i - a_{i-1})$. Now we will show the function $S_i(z)$ is Lipschitz with constant $O(a_i - a_{i-1})$. 

Since
   $ Dg_k(z)=W_k^{\!\top}\,D h(a_k W_k z)\,W_k, $
we have
\begin{align*}
D S_i(z) &= W_i^{\!\top}\,D h(a_i W_i z)\,W_i - W_{i-1}^{\!\top}\,D h(a_{i-1} W_{i-1} z)\,W_{i-1},\\
         &= W_i^{\!\top}\!\Bigl[D h(a_iW_i z)-D h(a_{i-1}W_{i-1} z)\Bigr]W_i
             \;+\; \\ 
         &+ \bigl(W_i^{\!\top}-W_{i-1}^{\!\top}\bigr)
                  D h(a_{i-1}W_{i-1} z)\,W_{i-1}\\
         &+\; W_i^{\!\top} D h(a_{i-1}W_{i-1} z)\bigl(W_i-W_{i-1}\bigr).
\end{align*}
Smoothness \eqref{eq:smoothness} and $\|z\|\le R$ give
\begin{align*}
   \|D h(a_iW_i z)-D h(a_{i-1}W_{i-1} z)\| &\;\le\;\alpha\bigl\|a_iW_i z-a_{i-1}W_{i-1} z\bigr\| < \alpha R \| a_i - a_{i-1}\| (C a_i + 1). \\
\end{align*}

Therefore, combining the bounds for the terms of $D S_i(z)$ and using the triangle inequality, we obtain
\begin{align*}
   \| D S_i(z) \| &\leq 
   \| Dh(a_iW_i z)-D h(a_{i-1}W_{i-1} z)\| +
   \|W_i^{\!\top}-W_{i-1}^{\!\top}\| \|D h(a_{i-1}W_{i-1} z)\| + \\
   &\quad\quad\quad\quad\quad\quad\quad\quad\quad\quad\quad\quad\quad\quad\quad  
   + \|Dh(a_{i-1}W_{i-1} z)\| \|W_i-W_{i-1}\| \\
   &\leq \alpha R \| a_i - a_{i-1}\| (C a_i + 1) + 2C \| a_i - a_{i-1} \| \|D h(a_{i-1}W_{i-1} z)\| \\
   &\leq \| a_i - a_{i-1} \| \left( \alpha R (C a_i + 1) + 2C \alpha a_{i-1} R  \right) = \\
   &\leq \alpha R \| a_i - a_{i-1} \| \left( C a_i + 2C a_{i-1} + 1 \right).
\end{align*}

Now we can bound the Lipschitz constant of $S_i$:
\begin{align*}
   \|S_i(v) - S_i(u)\| &\leq \|D S_i(z)\| \cdot \|v - u\| \\
   &\leq \alpha R \| a_i - a_{i-1} \| \left( C a_i + 2C a_{i-1} + 1 \right) \cdot \|v - u\| \\
   &= \alpha R M^{-1} \| a_i - a_{i-1} \| \left( C a_i + 2C a_{i-1} + 1 \right) \cdot \|x - y\| \\
\end{align*}

Going back to the difference we need to bound, we have
\begin{align*}
   \bigl\|\bigl(h_i(x) - W_i x\bigr) - \bigl(h_i(y) - W_i y\bigr)\bigr\| &\leq 
   \|S_i(u) - S_i(v)\| + \|I - W_i\| \cdot \|x - y\| \\
   &< \| x - y \| \| a_i - a_{i-1} \| \cdot \Bigl[ 
      \alpha R M^{-1} \left( C a_i + 2C a_{i-1} + 1 \right) + C \Bigr] \\
\end{align*}

\subsubsection{Combining the estimates}

We have shown that for all $i=1,\ldots,m$:

\begin{align*}
   \| h_i - W_i \|_L \leq  \| a_i - a_{i-1} \| \cdot \max \{ 
      &\tfrac{\alpha}{2} M^2 +  \alpha (C a_i + 1) R^2 +3C \left\| a_{i-1} M R \right\|, \\
      &\alpha R M^{-1} \left( C a_i + 2C a_{i-1} + 1 \right) + C \} \\
      \leq \| a_i - a_{i-1} \| \cdot \max \{
      &\tfrac{\alpha}{2} M^2 +  \alpha (C + 1) R^2 +3C M R, \\
      &\alpha R M^{-1} \left( 3C + 1 \right) + C \}.
\end{align*}

The choice of $a_i = (1-c)^{m-i}$ guarantees $a_i - a_{i-1} = O\left(\frac{\log m}{m}\right)$, 
which implies that $\| h_i - W_i \|_L = O\left(\frac{\log m}{m}\right)$ for all $i=1,\ldots,m$.  

Finally, we can conclude that the composition of $h_m\circ h_{m-1}\circ\cdots\circ h_1$ converges to $h$ in the sense of the theorem.

\end{proof}  

\subsection{Optimality Conditions via Zero Fréchet Derivatives}

\textit{NOTE: Again reference to \cite{bartlett2018representingsmoothfunctionscompositions}}

\subsection{Analyzing the convergence of orthogonal connections during training}

\textit{TODO: Refer to the papers where authors show singular vectors of consecutive layers converge to each other.}
  
\section{Empirical results}

\subsection{Performance}

\begin{figure}[H]
   \centering
   \begin{minipage}[c]{0.48\textwidth}
      \centering
      \includegraphics[width=\textwidth]{cifar10_resnet18_vs_orth_val.png}
   \end{minipage}
   \hfill
   \begin{minipage}[c]{0.48\textwidth}
      \centering
      \includegraphics[width=\textwidth]{cifar100_resnet18_vs_orth_val.png}
   \end{minipage}
   \caption{Comparison of validation accuracy of ResNets with identity and orthogonal skip connections. \emph{Left}: CIFAR-10. \emph{Right}: CIFAR-100.}
   \label{fig:cifar_comparison_plots}
\end{figure}

\begin{figure}[H]
   \centering
   \begin{minipage}[c]{0.48\textwidth}
      \centering
      \includegraphics[width=\textwidth]{residual_blocks_synthetic_comparison.png}
   \end{minipage}
   \caption{Comparison of validation accuracy of ResNets with identity and orthogonal skip connections on a synthetic dataset. \emph{Upper row}: Identity blocks. Learned non-linear transformations. \emph{Middle row}: Orthogonal blocks. Learned non-linear transformations. \emph{Lower row}: Orthogonal blocks. Learned orthogonal matrices.}
   \label{fig:synthetic_comparison_plots}
\end{figure}

In contrast to~\cite{wang2017orthogonalidempotenttransformationslearning}, using orthogonal skip connections instead of the identity on CIFAR-10, CIFAR-100 doesn't show any difference, as seen on Figure \ref{fig:cifar_comparison_plots}. 

Similarly for training on synthetic data, as seen on the left plot of Figure \ref{fig:synthetic_comparison_plots} \textit{Note: add a different plot here.}

\subsection{Validity of main theorem: distance of orthogonal weights to identity}

\begin{figure}[H]
   \centering
   \begin{minipage}[c]{0.48\textwidth}
      \centering
      \includegraphics[width=\textwidth]{cifar100_orth_resnet18_vs_34_id_conv.png}
   \end{minipage}
   \hfill
   \begin{minipage}[c]{0.48\textwidth}
      \centering
      \includegraphics[width=\textwidth]{cifar100_orth_resnet18_vs_34_id_conv.png}
   \end{minipage}
   \caption{Trajectory of orthogonal skip connections of ResNet18 and ResNet50. \emph{Left}: CIFAR-10. \emph{Right}: CIFAR-100. \textit{Note: add a different plot here.}}
   \label{fig:orthogonal_skip_connections_smoothness}
\end{figure}

\begin{figure}[H]
   \centering
   \begin{minipage}[c]{0.48\textwidth}
      \centering
      \includegraphics[width=\textwidth]{synthetic_orth_to_id.png}
   \end{minipage}
   \hfill
   \begin{minipage}[c]{0.48\textwidth}
      \centering
      \includegraphics[width=\textwidth]{synthetic_pred_dim2_nc2.png}
   \end{minipage}
   \caption{Trajectory of orthogonal skip connections on a synthetic dataset. \emph{Left}: Trajectories of different blocks. \emph{Right}: Example dataset and predictions.}
   \label{fig:orthogonal_skip_connections_smoothness}
\end{figure}

It appears that the main theorem doesn't hold as much

\subsection{Different optimizers}

% Bibliography
\bibliographystyle{plainnat}
\bibliography{references}

\end{document}


